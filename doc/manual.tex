\documentclass[12pt]{article}

\textheight 8.85in
\topmargin -1.4cm
\textwidth 6.6in
\oddsidemargin 0.1in
\evensidemargin 0in

\renewcommand{\figurename}{Figure}
\usepackage{graphicx, verbatim,hyperref,tcolorbox, amssymb, fancyhdr, lastpage}

\pagestyle{fancy}
\fancyhf{}
\rfoot{\center \thepage}
\renewcommand{\headrulewidth}{0pt}
\DeclareGraphicsRule{.JPG}{.eps}{*}{`jpeg2ps #1}

\begin{document}

\title{\LARGE {\sc matsml}: a machine-learning toolkit for materials science}
\author{{Huan Tran}\\ 
	{Georgia Institute of Technology}\\
\texttt{huan.tran@mse.gatech.edu}}

\maketitle
\tableofcontents

\section{Overview}
{\sc matsml} is a python-based machine-learning (ML) toolkit for some generic problems in materials science. Being initiated to support some of my (Huan Tran) non-polymer ML works, {\sc matsml} was designed to be {\it portable and self-contained}, providing necessary tools to follow the {\it entire} workflows reported, i.e., obtaining raw data, featurizing them, training models, and making predictions using the models developed. Given this objective, the materials used for my works can be found in some specific examples while other examples are for tutorial purposes and beyond. {\sc matsml} is freely available at \texttt{https://github.com/huantd/matsml.git}.

All computed data used in this toolkit are from my works. In cases that involve experimental data, when copyright and ownership prevent the sharing, suitable freely available alternative data are provided for demonstration purpose. Questions, requests, and comments on any parts of {\sc matsml} are welcome at \texttt{huan.tran@mse.gatech.edu} or \texttt{huantd@gmail.com}.

\section{Technical details}
\subsection{Code and installation}
{\sc matsml} is freely available at \texttt{https://github.com/huantd/matsml.git}. Dependencies needed for {\sc matsml} are
\begin{enumerate}
  \setlength\itemsep{-0.3em}
	\item \texttt{numpy}, version 
	\item \texttt{scikit-learn}, version 
	\item \texttt{keras}
	\item \texttt{tensorflow}
	\item \texttt{tensorflow\_probability}
	\item \texttt{matplotlib}
\end{enumerate}
Users are suggested to (1) create a devoted environment for this toolkit, e.g., using conda with python3, (2) install the dependencies, and (3) then install the source code of {\sc matsml} by
\begin{tcolorbox}
\begin{verbatim}
python3 setup.py install
\end{verbatim}
\end{tcolorbox}

\subsection{Functionalities}
In a typical workflow of materials informatics, one needs to (1) prepare/generate data, (2) featurize (or fingerprint) the data, (3) learning the featurized data to make models, and (4) using the developed models to make prediction on the featurized data of new cases. This toolkit supports all of these steps with an exception of (1). Because preparing or generating data is generally very time- and labor-consuming, some raw datasets are provided. More details on the data can be found in Sec. \ref{sec:data} of this manual. 

\section{Data}\label{sec:data}
\subsection{Sources}
Computed data used in this toolkit come from the following papers
\begin{enumerate}
  \setlength\itemsep{-0.3em}
    \item Vu Ngoc Tuoc and Tran Doan Huan, {\it Predicted binary compounds of tin and sulfur}, \href{https://pubs.acs.org/doi/abs/10.1021/acs.jpcc.8b04328}{J. Phys. Chem. C {\bf 122}, 17067 (2018)},
    \item Tran Doan Huan, {\it Pressure-stabilized binary compounds of magnesium and silicon}, \href{https://link.aps.org/doi/10.1103/PhysRevMaterials.2.023803}{Phys. Rev. Mater. {\bf 2}, 023803 (2018)},
   \item Chiho Kim, Tran Doan Huan, Sridevi Krishnan, and Rampi Ramprasad, {\it A hybrid organic-inorganic perovskite dataset}, \href{http://www.nature.com/articles/sdata201757}{Sci. Data {\bf 4}, 170057 (2017)}.
\end{enumerate}
For the purpose of using in this toolkit, these datasets were collected, formated, and provided as specified in the corresponding examples.

\subsection{Specifications}
Having raw data, the next step is to 


%\begin{figure}[ht]
%  \begin{center}
%  \includegraphics[width=0.65\linewidth]{/home/huan/work/papers/ml-perovskites3/figs/models/learning_curve-crop.pdf}
%  \caption{Learning curvers of ${\cal M}_1$, ${\cal M}_2$, ${\cal M}_3$, and ${\cal M}_4$. Each data point was obtained from 20 independent runs.}
%  \end{center}
%\end{figure}

\end{document}
